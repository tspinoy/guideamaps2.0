\chapter{User Classes \& Requirements}\label{ch:requirements}

In general, an application should meet a lot of requirements in order to deliver some quality to the users. A system should provide the right kind of functionality, but it should also be \textit{usable}, i.e. easy to learn and easy to use. Below are two definitions of usability which are often used:
\begin{description}
	\item[Definition 1] \hfill \\
	Usability is a measure of the ease with which a system can be learned and used, its safety, effectiveness and efficiency, and attitude of its users towards it. \hfill \citep{usability-definition-preece}

	\item[Definition 2] \hfill \\
	Usability is the extent to which a product can be used by specified users to achieve specified goals with effectiveness, efficiency and satisfaction in a specified context of use. \hfill \citep{usability-definition-improved}
\end{description}

Note that the second definition emphasizes the fact that usability is dependent on the target users and on the context of use. This means that one system can be usable for one type of user but not for another type of user or usability in one context but not in another context. Therefore, we also have to consider the different users of a system. In general, the users are classified into user classes. Users in a user class are similar in terms of their characteristics and how they use the system.\\

Hence, we can distinguish between functional requirements and usability requirements. The system should meet the functional requirements to provide the right functionality and it should meet the usability requirements to be usable. There are also requirements that do not belong to either category, e.g. the so-called non-functional requirements.\\

This chapter starts by identifying the different user classes. Next, we will present and justify the major requirements formulated for our system, functional as well as usability requirements and other requirements. Detailed requirements and requirements that are rather straightforward are omitted here. Later in this thesis, when we explain the implementation details, we will come back to these requirements and discuss how we managed to meet them.



% ----------------------------------
% ---------- USER CLASSES ----------
% ----------------------------------
\section{User Classes}\label{sec:user-classes}
In our system, each user can be classified in one of the following three user classes: (1) developer, (2) map creator and (3) end-user. The class the user belongs to defines its rights and possibilities, which are not particularly unique for the class. An overlap in the rights is possible, i.e. users of different classes can have a number of rights in common.

\subsection{Developer}\label{sec:user-class-developer}
A developer is a user who can define and implement a custom layout for the nodes and the links. This means that, if the purpose of the visualization differs from the default, a developer can make sure that the tool can be used with a different layout while the possibilities (in terms of functionality) are still the same. Developers are the only users that are able to do this because map creators and end-users are not allowed to extend the code with a custom implementation. The exact steps that need to be taken in order to provide a custom layout in the tool are explained in sections \ref{ch:implementation} and \ref{ch:usecase}.

\subsection{Map Creator}\label{sec:user-class-map-creator}
A map creator has the rights to change the structure of the visualization. With the structure, we don't mean the layout of the nodes and the links, but the composition of the map itself. Hence, the map creator ``initializes'' the map by creating the nodes and links necessary for his goal. He provides a title and eventually a description for each node. As a map creator, it is impossible to already provide the content of the node because that is the responsibility of the end-user. 

\subsection{End User}\label{sec:user-class-end-user}
The end-user is the most restricted user in terms of rights. However, it is this user that has to provide the most important part, i.e. the content of the nodes. They cannot change the title or the description, provided by the map creator because this would give them the opportunity to change the structure of the visualization (which is a right reserved for the map creator). If the map creator did not provide this information, the end-user can add it, but never change it again afterwards, because, once the information is added, we are in the same situation as when the map creator would have provided it.



% ----------------------------------
% ---------- REQUIREMENTS ----------
% ----------------------------------
\section{Requirements}\label{sec:requirements}

\subsection{Functional Requirements}\label{sec:functional-requirements}

% List of functional requirements
\begin{enumerate}[label=\textbf{\arabic*}., ref=\arabic*]
	
	\item \textbf{Customization\label{item:customization}} \hfill \\
	Goal \ref{item:goal:extend-modify} formulated earlier stated that the new version of GuideaMaps should allow the end-user to extend and modify the pre-defined map in some restricted way, i.e. they should be able to change the background color of nodes, to add child nodes and to remove nodes they don't need, and it should also be possible to add new options in choice nodes. However, end-users should only be able to delete nodes they added themselves and not the nodes defined by the map creator. Otherwise it would be possible for end users to change the pre-defined map completely. The purpose of allowing the end-user to edit the map in some restricted way is to adjust it to situations that were not foreseen by the map creator.
  
	\item \textbf{Modes And Rights\label{item:modes-rights}} \hfill \\
	Next to the developer, we have two user classes that require different functionality. Therefore, two modes are needed: and end-user mode and a map creator mode. These modes can be seen as a security mechanism to not let a particular end-user mess with a map definition. Depending on the mode the application is in, the corresponding functionality should be available. For example, in map creator mode, there should be a way to add a child node, while in the end-user mode, this functionality should not be available.
  
	\item \textbf{Zoom the visualization}\label{item:zoom} \hfill \\
	Zooming in or out such that you get less or more information at the same time on the screen is a frequently provided functionality for large visualizations. A feature to zoom is, for example, very useful in situations where you want to compare different parts of the visualization or focus on a certain part. The scrolling gesture is probably the best gesture for this action, because this is a well-known way to zoom in applications (e.g. Google Maps). Using the same gesture as in other applications will improve the learnability of our tool.
	
	\item \textbf{Zoom to fit\label{item:zoom-to-fit}} \hfill \\
	A variation on the zooming feature is ``zoom to fit'' (a.k.a. zoom until the complete figure fits into the bounding box). With custom zooming, the user can set the zooming level to meet its needs. Zoom to fit automatically adapts the zooming level and moves the content of the application until everything fits on the screen. This feature can, for example, be very useful to get an overview of the current data in the visualization.
		
	\item \textbf{Genericity\label{item:genericity}} \hfill \\
	Goal \ref{item:goal:generic} formulated earlier states that the application should be usable for different purposes. While GuideaMaps was usable in the context of domain specific requirement elicitation, our tool should also be useful in many other cases. Therefore, some requirements concerning the genericity of the application are needed:
  	\begin{enumerate}
		\item The tool should be generic in such a way that it is possible to use a different representation, e.g. shape/size for the nodes and the links.
		\item The default implementation should not be changed. An implementation for the nodes and the links created by a developer, should be \textit{plugged in} into the system, without affecting the default implementation.
	\end{enumerate}
	
	\item \textbf{Simultaneously Edit Same Visualization\label{item:work-simultaneously}} \hfill \\
	Multiple users should be able to work on the same visualization at the same time.
  
\end{enumerate}



\subsection{Usability Requirements}\label{sec:usability-requirements}

% List of usability requirements
\begin{enumerate}[label=\textbf{\arabic*}., ref=\arabic*]
	\item \textbf{Intuitiveness\label{item:intuitiveness}} \hfill \\
	It is important to keep in mind that the tool should not only be used by people with experience in Computer Science. It does not matter whether or not the user has a background in Computer Science, he should be able to easily learn to use the system in a short time. Therefore, design choices have to be made carefully. 
	
	\begin{enumerate}
	
		\item As icons\footnote{\url{https://fontawesome.com/}} take in general less space than text, we will use icons to indicate possible actions. Some examples of these actions are (1) adding a child node, (2) viewing the content of a node, (3) editing a node and (4) expand/collapse a node. However, it is important to choose for clear, not misunderstandable icons. The icons should not be ambiguous, they should link to one particular action and thus the user should know exactly what to expect when clicking on the icon. Well chosen icons are one of the factors in the design that contribute to learnability and ease of use.\\
	
		\item As already mentioned in Functional Requirement \ref{item:zoom}, gestures for common actions should not differ from the gestures used for the same action in other applications.
				
	\end{enumerate}
	
	\item \textbf{Learnability\label{item:learnability}} \hfill \\
	\textcolor{red}{TODO: Dit is geen echt usability requirement. Moet beter.}\\
	Intuitiveness and learnability are somehow related to each other. It is easier to learn how a system works if the possibilities are intuitive and actions are not hidden.
	
	\item \textbf{Efficiency\label{item:efficiency}} \hfill \\
	\textcolor{red}{TODO: Dit is geen echt usability requirement. Moet beter.}\\
	As a user, it should be easy to achieve your goals. The application should be created in such a way that user errors are (almost) impossible.
	
	\item \textbf{Accessibility\label{item:accessibility}} \hfill \\
	The target audience of the application should be limited as little as possible. 
	\begin{enumerate}
		\item We don't expect the user to have a background in Computer Science or visualization techniques. \textcolor{red}{TODO: Dit is geen echt usability requirement. Moet beter.}
		\item Color blind people should still be able to use the system.
	\end{enumerate}
	
\end{enumerate}



\subsection{Other Requirements}\label{sec:other-requirements}

\begin{enumerate}[label=\textbf{\arabic*}., ref=\arabic*]
	\item \textbf{Device- and OS-independent\label{item:device-os-independent}} \hfill \\
	According to Research Goal \ref{item:goal:device-os-independent}, the application should run on different kinds of devices (e.g. tablets, laptops and desktops) and operating systems (Android, iOS, MacOS and Windows). While version 1.0 was only designed for an iPad, our version should provide a solution for this restriction.\\
	
	The only restriction on the used device is that it needs to have a screen that is large enough because it is not very convenient to work with the visualization on small screen areas, e.g. on smartphones. The application could run on smartphones but it is not recommended nor required to use it on devices with relatively small screens.
	
	\item \textbf{Code Separation\label{item:code-separation}} \hfill \\
	The core of the application should be completely separated from custom implementations created by developers.

\end{enumerate}


