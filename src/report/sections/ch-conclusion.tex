\chapter{Conclusion \& Future Work}\label{ch:conclusion-future-work}

\section{Conclusion}\label{sec:conclusion}

\section{Future Work}\label{sec:future-work}

\textcolor{blue}{
The presented application is not perfect yet and some improvements could be done in the future. One of the most important improvements that could be done is supporting a broader range of browsers. Currently, the tool works perfectly in the latest version of Google Chrome. It is good to have a working tool in one of the most widely used browsers, but lots of people use other browsers apart from Google Chrome. We detected some issues in Firefox, where for instance the tool is very slow on a tablet. In the future, we should improve the system such that it works in different browsers and if possible in all commonly used ones.}\\

\textcolor{blue}{
Further, it is also important to be able to save the changes made on the maps. Therefore, the system should be further elaborated so that the changes are written to a file which is loaded the next time the map is opened. Right now, the maps are re-initialized every time the page is refreshed.}\\

\textcolor{blue}{
Being able to save the changes also gives the occasion to start to work on a way to make the tool collaborative, i.e. multiple people can simultaneously work on the same map and see each others (saved) changes in real-time.\\
}

\textcolor{blue}{
Another important improvement would be to make it possible to drag and drop the nodes of the visualization to support the similarity and proximity principle of the Gestalt Psychology Theory \citep{koffka2013principles}.\\
}
