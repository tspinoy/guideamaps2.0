\section*{Abstract}%\enlargethispage{1.5\baselineskip}
\markboth{Abstract}{Abstract} % ensure the correct name appears in the header and footer

\color{blue}

Creating visualizations of linked data is a feature present in many tools nowadays. Unfortunately, most of them lack other important functionality, like being able to define your own layout for your visualization, to create templates or to do more than just creating a mind map summarizing your thoughts. This thesis describes \textit{GuideaMaps 2.0}, a browser-based tool to create knowledge maps in a convenient way and which does not lack the previously mentioned functionality. With convenient we mean: not being limited to a particular device or platform, being able to use the tool for several purposes, etc. While a first version of GuideaMaps was mainly created for the purpose of serious games, we broadened the range of situations in which the tool can be used. As an example use case, a website with a complicated tree structure underneath was visualized by our tool to make its structure more clear. In a user study with more than 40 participants we found our visualization was easier to use.\\

Another contribution is that the system is created as a library, i.e. developers can extend and modify the implementation of the tool if this is needed for their goal. The library allows to customize the layout of the nodes and the links without affecting the default implementation of the tool. A custom implementation can easily be plugged in. Further, two modes are configured: (1) map creators can create templates for a specific goal and (2) end-users can fill the templates with the necessary data. Hence, GuideaMaps 2.0 is different than the tools we find on the internet in many ways and other tools are often usable for a single purpose, while our application tries to be functional in many situations.

\color{black}