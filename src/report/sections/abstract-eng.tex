\section*{Abstract}%\enlargethispage{1.5\baselineskip}
\markboth{Abstract}{Abstract} % ensure the correct name appears in the header and footer

Many tools nowadays support the creation of visualizations of linked data, e.g. mind map tools. Unfortunately, most of these tools lack other important functionality, like being able to define your own look and feel for your visualization, to create templates, or to do more than just creating a mind map summarizing your thoughts. This thesis presents \textit{GuideaMaps 2.0}, a browser-based tool to create knowledge maps in a convenient way and which does not lack the previously mentioned functionality. With convenient we mean: not being limited to a particular device or platform and being able to use the tool for several purposes. While a first version of GuideaMaps was mainly created for the purpose of requirement elicitation, we broadened the range of situations in which the tool can be used. As an example use case, a website with a complicated tree structure underneath was visualized by our tool to make its structure more clear. In a user study with 52 participants evaluating this use case we found our visualization was easier to use than the original website.\\

Another contribution is that the system is created as a library, i.e. developers can extend and modify the implementation of the tool if this is needed for their goal. The library allows to customize the visualization of the nodes and the links without affecting the default implementation of the tool. A custom implementation can easily be plugged in. Further, two modes are foreseen: (1) map creators can create templates for a specific goal and (2) end-users can fill the templates with the necessary data. Hence, GuideaMaps 2.0 is different than the existing tools in many ways. Other tools are often usable for a single purpose, while our application tries to be functional in many situations.\\

The standard GuideaMaps visualization was evaluated with another user study. The results showed that creating such a tool is not straightforward. Small details (e.g. icons) can lead to frustration and make the tool less intuitive.