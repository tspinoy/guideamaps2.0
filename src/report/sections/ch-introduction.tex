\chapter{Introduction}\label{ch:introduction}

The more people can store in their memory, the less they have to look up in external sources and the faster they can work. However, processing and storing a big amount of information in the human memory for later reuse is not easy, especially when the information is presented in the form of a long text. Therefore, students for example, make schemes and summaries of their study material. The reason why they do this is to make it easier to learn and remember the material. Schemes and visualizations make relations between pieces of information explicit and are easier to grasp. In addition, most people are visual learners, meaning that they learn better when using visual aids. Not only the way of learning new subject material, but also other activities can be supported by means of a visualization. For instance, if you write down the structure of a computer program in words, it is more difficult to discuss that structure with other people than when the structure is expressed by means of a diagram. This can be explained as follows.\\

According to \cite{moody-physicsofnotations-2009}, there exists a difference between visual notations and textual languages in how they encode information and how they are processed by the human mind: textual representations are one-dimensional and are processed serially by the auditory system, while visual representations are two-dimensional and processed in parallel by the visual system.\\

Further, the form of representations/visualizations has a greater effect on understanding and problem solving than their content \citep{moody-innsbruck-2012}. In other words, for a visualization, the way the content is represented is more important than the content itself.\\

\cite{moody-physicsofnotations-2009} also defines the term \textit{Cognitive effectiveness} as ``the speed, ease and accuracy with which a representation can be processed by the human mind''. The better the cognitive effectiveness of a visual notation, the better human communication and problem solving can be done.\\

The goal of this thesis is to create a tool with high cognitive effectiveness, in which it is possible to represent linked data and knowledge in a visual manner. It should not only be possible to create a visual notation of the data, but also to edit the data as well as extending the representation with additional data. The solution is based on a visualization tool created by \cite{erikjanssens}, called GuideaMaps. This application was mainly built to provide support for the requirement elicitation for serious games. It provides the functionality to enter information in pre-defined maps (trees of nodes). More details about this application can be found in chapter \ref{ch:background-guideamaps}. We present a new version of GuideaMaps, which can be used for other purposes than serious games and which has a bunch of interesting improvements under the hood.





\section{Problem Statement}\label{sec:problem-statement}
For the first version of GuideaMaps, the main goal was to develop the following:

\begin{quote}
``A tool that allows the different people (and with different background) involved in the development of a serious game (e.g., against cyber bullying) to brood over the goals, characteristics and main principles of a new to develop serious game. The tool should be easy to use and usable in meetings. Therefore, we want to explore the characteristics and capabilities of a tablet (i.e. iPad).'' \hfill \citep{erikjanssens}
\end{quote}

By specifying the goal in this way, end users of the application are restricted in different ways. First, they need an iPad to be able to use the application. Another type of tablet with a different operating system is not possible, because GuideaMaps was created and designed for iOS only. If someone doesn't have access to an iPad, (s)he cannot use the application, which is a hard restriction.\\

Furthermore, the tool focuses on requirement elicitation. Initially, the tool was created for the purpose of requirement elicitation for serious games, but it can also be used for the requirement elicitation in other domains \citep{detroyerjanssens}. However, because the visualizations are based on pre-defined templates, the nodes can only be edited by the end-user in a limited way: content can be given and the background color can be changed, but the end-user cannot add new nodes. In addition, the visual notation used is fixed: the creators of the visualization templates cannot edit the representation of a node or define their own representation (e.g. change the length and width or use a different shape). Not being able to do this is a limitation in the sense that for some purposes this default visualization may not be very suitable. Furthermore, for defining a template the author had to use XML and no graphical editor was available for this purpose making it harder for non-ICT schooled people to define new templates.





\section{Research Goals}\label{sec:research-goals}
The issues discussed in the previous section indicate that the first version of GuideaMaps comes along with some limitations. Therefore, the following research goals for a new version of GuideaMaps were formulated:

\begin{enumerate}[label={\textcolor{white}{\arabic*}},ref=\arabic*]
	\item \hspace*{-2.5em}\textbf{Goal 1\label{item:goal:device-os-independent}} \hfill \\
	The new version of GuideaMaps should work on all common devices and on different operating systems.
	
	\item \hspace*{-2.5em}\textbf{Goal 2\label{item:goal:templates}} \hfill \\
	It should be possible to pre-define the maps, i.e. the templates, in a graphical way.
	
	\item \hspace*{-2.5em}\textbf{Goal 3\label{item:goal:extend-modify}} \hfill \\
	The new version of GuideaMaps should allow the end-user to extend and modify the pre-defined map in some restricted way.
	
	\item \hspace*{-2.5em}\textbf{Goal 4\label{item:goal:generic}} \hfill \\
	The application should be generic in such a way that it can be customized to be usable for different purposes, i.e. the user should be able to define its own graphical representation for the visualization.
\end{enumerate}

This thesis presents a new version of GuideaMaps, called \textit{GuideaMaps 2.0}, taking the research goals into account. How the tool achieved the formulated research goals is explained into detail in the rest of this thesis.\\




\section{Thesis Structure}\label{sec:thesis-structure}
In the next chapter, we start with a brief explanation of the first version of GuideaMaps. Because our solution is mainly based on this application, we start that chapter with a discussion of the different concepts and principles used. We advise the reader to go through this chapter if (s)he is not familiar with GuideaMaps as the rest of this thesis will refer to these concepts and principles.\\

Chapter \ref{ch:related-work}, called ``Related Work'', discusses different visualization techniques and existing tools for similar purposes.\\

Chapter \ref{ch:requirements} defines the different user classes for the system and the main requirements for the system divided into three categories, i.e. functional, usability and other requirements.\\

These two chapters are followed by a chapter on the implementation of our tool. Information about choices made and which technologies are used can be found in this chapter. Furthermore, we explain how GuideaMaps is implemented and how the code is structured as a library so that this library can also be used for other purposes.\\ 

In chapter \ref{ch:usecase}, we demonstrate the use of the library for another use case, demonstrating the achievement of Research Goal \ref{item:goal:generic}.\\

The work developed has also been evaluated by means of a user experiment. The approach and the results of the evaluation can be found in chapter \ref{ch:evaluation}.\\

In the last chapter, the conclusions of this thesis are presented as well as possible future work.



